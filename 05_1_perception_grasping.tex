Given an RGB-D image of the source bin and a CAD model of the object, the objective is to retrieve ($\object^i$, $\pose^i$) such that it maximizes the chance of achieving target configuration $\hat{p}^j$, where $D(p^i,\hat{p}^j) \leq \epsilon$. To achieve this, the image is passed through a {\tt MaskRCNN} convolutional neural network~\cite{he2017mask}, which is trained to perform segmentation and retrieve the set of object instances $\objectset$.  An image segment is ignored if it has a number of pixels below a threshold. It is also ignored, if {\tt MaskRCNN} has small confidence that the segment corresponds to the target object. Among the remaining segments, instances $\object^i \in \objectset$ are arranged in a descending order of the mean global Z-coordinate of all the RGBD pixels in the corresponding segment. Then, 6D pose estimation is performed for the selected instance~\cite{175}\cite{stocs}. 

If, given the detected 6D pose of the instance, the top-facing surface does not allow the placement of the object via a top-down pick, the next segment instance is evaluated in order of the mean global Z-coordinate. If no object reveals a top-facing surface, then the first object in terms of the maximum mean global Z-coordinate is chosen for picking. 

For the selected object, a picking point, i.e., a point where the suction cup will be attached to the object, is computed over the set of points registered against the object model. It utilizes a picking-score associated in a pre-processing step with each model point, which indicates the stability of the pick on the object's surface. The score calculates the distance to the center of the object mesh. A continuous neighborhood of planar pickable points is required to make proper contact between the suction cup and the object surface. Thus, a local search is performed around the best pick-score point to maximize the pickable surface.


% For the arm to pick $\object^i$ at $\pose^i$, it has to be that the arm's end-effector pose satisfies certain conditions relative to the object's pose as expressed by a binary output function: $\mathtt{is\_pick\_feasible}( \object^i, \pose^i, \grasp), \textrm{ where } \grasp \in \mathcal{T}.$ For instance, the pose $\grasp$ of a suction cup must align it with at least one of the surfaces of an object $\object^i$ at $\pose^i$. Then, it is possible to define the set of end-effector poses, which allow to pick an object at a specific pose: \vspace{-.05in}$$\graspset( \object^i, \pose^i ) = \{\grasp \in \mathcal{T}:  {\mathtt{is\_pick\_feasible}}(\object^i, \pose^i, \grasp) = true\}. \vspace{-.05in}$$ 

% Assume the sets $\graspset( \object_i, \pose_i )$